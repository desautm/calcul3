%% Pour compiler, on utilise XeLateX avec Ctrl-Shift-F1
%% �a produit une image PNG

\documentclass[convert={ghostscript, density = 1000}]{standalone}

\usepackage{pstricks-add}

\begin{document}

\psset{algebraic,unit=1}
\begin{pspicture}(-2,-0.5)(4,3.5)	

\psaxes[labels=none,ticks=none,arrowscale= 1.5]{->}(0,0)(-2,-0.5)(4,3.5)
\uput{3.5}[50]{0}(0,0){$r=f(\theta)$}
\Polar

\pscurve[fillstyle=solid,fillcolor=lightgray,linecolor=lightgray](3,22)(2.9,40)(3.2,50)(3.3,65)(2.9,80)(2.7,100)(3,110)(3.1,130)
\psline[fillstyle=solid,fillcolor=lightgray,linecolor=lightgray](0,0)(3,22)(3.1,130)(0,0)
\pscurve[linewidth=2pt](3,22)(2.9,40)(3.2,50)(3.3,65)(2.9,80)(2.7,100)(3,110)(3.1,130)
\psline[linewidth=2pt](0,0)(3,22)
\psline[linewidth=2pt](3.1,130)(0,0)
\psarc[arcsepB=4pt]{-}{1.5}{0}{22}
\psarc[arcsepB=4pt]{-}{0.75}{0}{130}
\uput{1.7}[11]{0}(0,0){$\theta_a$}
\uput{0.95}[70]{0}(0,0){$\theta_b$}
	
\end{pspicture}

\end{document}