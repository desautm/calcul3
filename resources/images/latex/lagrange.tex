%% Pour compiler, on utilise XeLateX avec Ctrl-Shift-F1
%% ?a produit une image PNG

\documentclass[convert={ghostscript, density = 1000}]{standalone}

\usepackage{pstricks-add}
\usepackage{fp}

\begin{document}

\psset{algebraic,unit=1}
\begin{pspicture}(-0.5,-0.5)(1.25,5.5)
	
\psset{algebraic,unit=0.75}
\pspicture(-0.5,-0.5)(7,5.5)
\psaxes[labels=none,ticks=none]{->}(0,0)(-0.5,-0.5)(7,5.5)[$x$,-90][$y$,180]
\multido{\i=1+1}{4}{%
	\psarc(5,5){\i}{180}{270}
	\FPeval{y}{5-\i}
	\FPeval{z}{5+\i}
	\FPtrunc{\z}{\z}{0}
	\uput[0](5,\y){$f(x,y)=\z$}
}
\pscurve(-0.5,2)(0,2)(2,3)(2.9,2.9)(2.8,2)(3.5,0)(3.8,-0.5)
\psdot[dotsize=0.1](2.9,2.9)
\uput[0](0,0.5){$g(x,y)=k$}
\psline{->}(0.5,0.8)(0.5,2)
\psplotTangent[linecolor=lightgray]{2.9}{2.9}{5-sqrt(9-(x-5)^2)}
\endpspicture
	
\end{pspicture}

\end{document}