%% Pour compiler, on utilise XeLateX avec Ctrl-Shift-F1
%% �a produit une image PNG

\documentclass[convert={ghostscript, density = 1000}]{standalone}

\usepackage{pstricks-add}
\usepackage{fp}

\begin{document}

\psset{unit=1,algebraic}
	\begin{pspicture}(-1,-1)(7,5)
		
		\psaxes[labels=none,ticks=none]{->}(0,0)(-1,-1)(7,5)
		
		\multido{\nbA=1+0.5}{10}{%
			\multido{\nbB=1+0.5}{6}{%
				\FPeval{nbAa}{\nbA+0.5}
				\FPeval{nbBb}{\nbB+0.5}
				\psframe[fillstyle=solid,fillcolor=lightgray](\nbA,\nbB)(\nbAa,\nbBb)
			}%
		}
		
		\psline[linestyle=dashed](1,0)(1,1) \uput[-90](1,0){$a$}
		\psline[linestyle=dashed](2.5,0)(2.5,1) \uput[-100](2.5,0){$x_{i-1}$}
		\psline[linestyle=dashed](3,0)(3,1) \uput[-80](3,0){$x_i$}
		\psline[linestyle=dashed](6,0)(6,1) \uput[-90](6,0){$b$}
		
		\psline[linestyle=dashed](0,1)(1,1) \uput[180](0,1){$c$}
		\psline[linestyle=dashed](0,2.5)(1,2.5) \uput[180](0,2.5){$y_{j-1}$}
		\psline[linestyle=dashed](0,3)(1,3) \uput[180](0,3){$y_j$}
		\psline[linestyle=dashed](0,4)(1,4) \uput[180](0,4){$d$}
		
		\psdot[dotsize=0.2](2.75,2.75)
		\uput[0](3.5,4.5){$(x^*_i,y^*_j)$}
		\psline{->}(3.5,4.5)(2.85,2.85)
		
	\end{pspicture}

\end{document}