%% Pour compiler, on utilise XeLateX avec Ctrl-Shift-F1
%% �a produit une image PNG

\documentclass[convert={ghostscript, density = 1000}]{standalone}

\usepackage{pstricks-add}
\usepackage{pst-solides3d}
\usepackage{fp}

\begin{document}

\psset{unit=1,algebraic}
	\begin{pspicture}(-1,-1)(5.5,5.5)
		
		\psaxes[labels=none,ticks=none]{->}(0,0)(-1,-1)(5,5)[$x$,-90][$y$,180]
	
		\pscustom{
			\pscurve(2,1)(1,2)(0.7,3)(2.3,4)
			\gsave
			\psline(2.3,4)(4,4)
			\pscurve(4,4)(4.3,3)(3.6,2)(3.2,1)
			\psline(3.2,1)(2,1)
			\fill[fillstyle=solid, fillcolor=lightgray]
			\grestore
		}
		\pscurve[linewidth=1.5pt](2,1)(1,2)(0.7,3)(2.3,4)
		\pscurve[linewidth=1.5pt](4,4)(4.3,3)(3.6,2)(3.2,1)
		
		\psline[linestyle=dashed](0,1)(3.2,1)
		\psline[linestyle=dashed](0,4)(4,4)
		
		\uput[0](2.1,2.5){$D$}
		
		\uput[180](0,1){$c$}
		\uput[180](0,4){$d$}
		
		\uput{0.25}[180](1.5,1.5){$x_1(y)$}
		\uput[0](3.6,2){$x_2(y)$}
		
	\end{pspicture}
\end{document}